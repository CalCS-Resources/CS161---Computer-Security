\documentclass{article}
\usepackage[utf8]{inputenc}
\begin{document}
Merkle Tree
\section{Bitcoin}
\begin{enumerate}
\item Best policy: ever spend your bitcoin
\item A currency where the best policy is to never spend
\item If you buy bitcoin you have to wait a few days before you can tell your bank it was a fraud and get your money back
\item Although the transaction itself is cheap, waiting time of ten minutes has real world cost
\item Bitcoin has no tie to dollars so it fluctuates up and down
\item Every actual bitcoin transaction: turning dollars to ditcoing, doing bitcoin, turning to back to dollars
\item Using bitcoin is a political statement and is censorship resistant (drugs)
\item Silk Road was a TOR hidden service marketplace - US centric - selling drugs and accepted bitcoin - Mandatory feedback and escrow system, optional currency hedge for sellers
\begin{enumerate}
\item The Armory: you could buy guns too - amazingly illegal for everyone - black market prices - 95 percent of people just buy guns legally
\end{enumerate}
\item Ransomeware: you infect a system, encrypt files, encrypt master key with extortionist's public key, tell them to pay money or they won't see there file again - hard to get paid
\item Can't store Bitcoin on an Internet connected device - or you will get robbed - store with a private key offline
\item The ECDSA signature scheme has a nonce (k-value) like r in El Gamal. If you sign two different messages with the same k, it becomes trivial to find the private key
\begin{enumerate}
\item Green Dot MoneyPak: service that allows you to ransfer money to reload prepaid credit cards for 6 dollars - this used to be preferred by ransomware but not Bitcoin
\item Bitcoin uses a lot of energy -more than UC Berkeley
\item Bitcoin is pseudonymous - every wallet is a distinct pseudonym and every transaction is public
\item Ethereum: "Smart" Contracts in a JavaScript-like Language
\begin{enumerate}
\item Executes a small virtual machine - Paymaent to a destination can invoke the destination's program
\item In paying someone else, it invokes code outside itself
\item At the same time, the cryptocurrency community tends to belive code is law - the basis of some very interesting attacks:
\begin{enumerate}
\item Denial of Service (DOS): Find code that is cheap in terms of "gas" but expensive in practice - like dis I/O - spend a bunch of transactions to slow down system and stop the network
\item Distibuted Autonomous Organazation: like a mutual fund whose investments are the consensus of the participants - including the ability to split off and perform other actions - it tended to be a bit of "natural ponzi" scheme right from the start - nearly 10 percent of all Ethereum was invested in the DAO \\ 
Bug: Attacker could propose a split that he gets all his money - split process would first transfer money the reduce the balance - in transferring the destination is simply calling another function, one being written by the attacker - the attacker can claim he followed the contract (the code) - classic TOCTOU
\\ This caused the Ethereum community to split in two - one group said "revise history" and simply have the miners ignore the DAO theft - the other group kept the old chain going because they didn't have their money stolen from them

\end{enumerate}
\end{enumerate}
\end{enumerate}
\item Anything electronic must have an undo - detection and mitigation
\item Bitcoin has a limited amount of programmibility - you can pay two an address - inputs are actually small scripts
\item Allowed to set up a structure where in order for some transaction 2 or 3 signuatures must be present
\end{enumerate}
\subsection{How to Make Money in Bitcoin in 10 Easy Steps}
\begin{enumerate}
\item Move to Sochi
\item Break into blockchain.info and other web-wallet services
\item Download saved web wallets for offline cracking
\item Modify wallet service javascript to leak passwords
\item Be patient and wait
\item When discovered, steal all the Bitcoin
\item Blame the victims
\item Write malcode to look for Bitcoin wallets
\item Crack away and rob everyone
\item Enjoy life
\end{enumerate}




\section{Key Management}
Problem with sending a public key: MITM can change to be his own public key and decry pt any message.
\subsection{PKI - Public Key infrastructure }
Helps manage public keys and certificates
\subsubsection{Trusted Directory}
\begin{enumerate}
\item \textbf{Trusted Directory} contains mapping of User to User Public key that cant be interfered with. Trusted Directory has a secret key and public key $PK_{TD}$ and trusted directory is accessible by anyone. 
\item Trusted Directory signs transmission of data so receiving users can check if there was a Man in the Middle Attack. $Sig[SK_{TD}, PK_B]$. However there is still an issues: MITM can get give Alice his/her own public key and pretend to be bob by passing on Trusted Directory Data to Alice (Alice thinks this is safe because it was 'signed' by TD, however MITM could have sent alice his). To get around this, have TD send the following 'BOBS PUBLIC KEY IS $PK_B$'. Thus the TD includes data on whose key they are signing and alice can confirm whose key they receive.
\item MITM requests Eve's public key, and TD answers with $Sig[SK_{TD}, PK_M]$. Then Eve can send this back to Alice instead of $Sig[SK_{TD}, PK_B]$ Instead, send $Sig[SK_{TD}, \textrm{Bob's public key is } PK_B]$
\item \textbf{Replay Attack}: Bob changes keys, so he know has $SK_B'$ and $PK_B'$. We imagine the MITM wants to intercept the TD saying $PK_B'$ and replace it with $PK_B$. Alice must be able to determine it is the latest so we put a time stamp. 
\begin{enumerate}
\item Alice says she wants $PK_B$ and gives a nonce which she wants them to use. TD send back $Sig[SK_{TD}, \textrm{Bob's public key is } PK_B, nonce]$ This nonce shows the message was generated now, not earlier.
\end{enumerate} 
\item Caveats: central point of attack (and failure), Scale-ability: it has everyone's PK, important to authenticate users to TD when updating PK, the service has to be online at all times during PK operation because you must fetch PK
\end{enumerate}

\subsubsection{Digital Certificates}
\begin{enumerate}
\item \textbf{Certificate}: a way to represent association between name and PK as certified by a third party (CA = certificate authority), like Verisign
\item Using RSA: $Sign(SK_{CA},\textrm{name=PK})$
\item Alice obtains the certificate for $PK_B$ from anyone and she knows she can trust it. This means it doesn't have to be online at all times (one advantage)
\item Scale-ability also still a problem, although central point of failure no longer is
\item \textbf{Certificate chains or hierarchical PKI}: The Gov. of CA Jerry can maintain around 100 certificates, including UCB president, who has his own 100 in turn, including Popa, etc. $Sign(SK_J, Presidential=PK_P)$
\item If Alice wants to send a message to Nick - she asks people for certificate of Nick - then she gets back $Sign(SK_{UCB President}, Nick=PK_N)$. Then she must get the $Sign(SK_J, Presidential=PK_P)$ to verify the UCB president. Finally, everyone has hard-coded $PK_J$, so she can verify him as well. Now she knows that she received Nick's PK.
\end{enumerate}

\subsubsection{Revoking Certificate}: 
\begin{enumerate}
\item $Sign(SK_{CA}, PK_B=Bob\textrm{Experices on Nov 29, 2016}$ Checks that the client is still valid
\item Use revocation lists: browsers put together of no longer valid keys that did not expire yet - black list
\end{enumerate}


\end{document}

